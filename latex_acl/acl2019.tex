%
% File acl2019.tex
%
%% Based on the style files for ACL 2018, NAACL 2018/19, which were
%% Based on the style files for ACL-2015, with some improvements
%%  taken from the NAACL-2016 style
%% Based on the style files for ACL-2014, which were, in turn,
%% based on ACL-2013, ACL-2012, ACL-2011, ACL-2010, ACL-IJCNLP-2009,
%% EACL-2009, IJCNLP-2008...
%% Based on the style files for EACL 2006 by 
%%e.agirre@ehu.es or Sergi.Balari@uab.es
%% and that of ACL 08 by Joakim Nivre and Noah Smith

\documentclass[11pt,a4paper]{article}
\usepackage[hyperref]{acl2019}
\usepackage{times}
\usepackage{latexsym}
\usepackage[utf8]{inputenc}
\usepackage[russian,english]{babel}

\usepackage{url}

%\aclfinalcopy % Uncomment this line for the final submission
%\def\aclpaperid{***} %  Enter the acl Paper ID here

%\setlength\titlebox{5cm}
% You can expand the titlebox if you need extra space
% to show all the authors. Please do not make the titlebox
% smaller than 5cm (the original size); we will check this
% in the camera-ready version and ask you to change it back.

\newcommand\BibTeX{B\textsc{ib}\TeX}

\title{Automated Ontology Matching using Cross-lingual Embeddings}

\author{Denis Gordeev, Alexey Rey, Dmitry Shagarov
	First Author \\
  Affiliation / Address line 1 \\
  Affiliation / Address line 2 \\
  Affiliation / Address line 3 \\
  \texttt{email@domain} \\\And
  Second Author \\
  Affiliation / Address line 1 \\
  Affiliation / Address line 2 \\
  Affiliation / Address line 3 \\
  \texttt{email@domain} \\}

\date{}

\begin{document}
\maketitle
\begin{abstract}
  This document contains the instructions for preparing a camera-ready
  manuscript for the proceedings of ACL 2019. The document itself
  conforms to its own specifications, and is therefore an example of
  what your manuscript should look like. These instructions should be
  used for both papers submitted for review and for final versions of
  accepted papers.  Authors are asked to conform to all the directions
  reported in this document.
\end{abstract}


\section{Introduction}
\foreignlanguage{russian}{Это исторические статьи смежной тематики}
There are numerous structured information representations containing texts. Among them we can name ontologies, taxonomies, lexical databases such as WordNet. Most of them exist only for English. Many researchers have tried automatically converting such resources into their languages. Mostly attempts were focused on using machine translation engines. [] !!!!!! With the increase in popularity of word embeddings [] !!!!!! some researchers proposed cross-lingual word embedding models. However, most of the early works in this domain relied on massive parallel corpora. In 2017 

\section{Related work}
\foreignlanguage{russian}{Это статьи на ту же тему.}
\section{Methods}
\subsection{Representing ontologies as embeddings}
\subsection{Graph Matching}
\subsubsection{Hungarian}
\subsubsection{Greedy}
\section{Data}
\subsection{Taxonomies}
\textbf{NIGP}

\textbf{OKPD2}
\subsection{WordNet}
\section{Experiments}
\foreignlanguage{russian}{разные синсеты одного слова можно заматчить с помощью иерархической информации}
\section{Results}
\subsection{Taxonomy matching}
\section{Discussion}
\section{Conclusion}


\end{document}
